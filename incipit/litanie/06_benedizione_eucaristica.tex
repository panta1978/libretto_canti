\titolo{Benedizione eucaristica}

\vspace{0.25em}

\cantico{Genti tutte}

\setcounter{numstrofa}{0}

\strofa Genti tutte proclamate il mistero del Signor,\\
del suo Corpo e del suo Sangue che la Vergine donò,\\
e fu sparso in sacrificio per salvar l'umanità.

\spazio

\strofa Dato a noi da madre pura, per noi tutti s'incarnò.\\
La feconda sua parola tra le genti seminò.\\
Con amore generoso la sua vita consumò.

\spazio

\strofa Nella notte della Cena coi fratelli si trovò.\\
Del pasquale sacro rito ogni regola compì,\\
e agli Apostoli ammirati come cibo si donò.

\spazio

\strofa La parola del Signore pane e vino trasformò;\\
pane in carne, vino in sangue, in memoria consacrò.\\
Non i sensi, ma la fede, prova questa verità.

\spazio

\strofa Adoriamo il Sacramento che Dio Padre ci donò,\\
nuovo patto, nuovo rito, nella fede si compì.\\
Al mistero è fondamento la parola di Gesù.

\spazio

\strofa Gloria al Padre onnipotente, gloria al Figlio Redentor,\\
lode grande, sommo onore, all'eterna Carità.\\
Gloria immensa, eterno amore, alla Santa Trinità. Amen.

\medskip

\textbf{S:} Ci hai dato, Signore, il pane disceso dal cielo\\
\emph{\textbf{A:} che porta in sé ogni dolcezza.}

\medskip

\textbf{S:} Preghiamo: Signore Gesú, che nel mirabile sacramento dell'Eucarestia ci hai lasciato il memoriale della Tua Pasqua, fa che adoriamo con viva fede il Santo mistero del tuo Corpo e del tuo Sangue, per sentire sempre in noi i benefici della Redenzione.

Tu che vivi e regni nei secoli dei secoli.

\medskip

\emph{\textbf{A:}} Amen.

\medskip

\emph{Nel tempo di Pasqua}

\medskip

\textbf{S:} Preghiamo: infondi, Signore, il tuo Spirito di amore su tutti coloro che hai nutrito con i sacramenti pasquali. Per Cristo nostro Signore.

\medskip

\emph{\textbf{A:}} Amen.

\newpage

\textbf{\large{Invocazioni:}}

\medskip

Dio sia benedetto.\\
Benedetto il suo santo Nome.\\
Benedetto Gesù Cristo, vero Dio e vero uomo.\\
Benedetto il nome di Gesù.\\
Benedetto il suo sacratissimo Cuore.\\
Benedetto il suo preziosissimo Sangue.\\
Benedetto Gesù nel santissimo Sacramento dell'altare.\\
Benedetto lo Spirito Santo Paraclito.\\
Benedetta la gran Madre di Dio, Maria santissima.\\
Benedetta la sua santa e immacolata concezione.\\
Benedetta la sua gloriosa assunzione.\\
Benedetto il nome di Maria, vergine e madre.\\
Benedetto san Giuseppe, suo castissimo sposo.\\
Benedetto Dio nei suoi angeli e nei suoi santi.
