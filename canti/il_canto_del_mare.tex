\canto{Il canto del mare}

\rit{Cantiamo al Signore, stupenda è la sua vittoria.\\
Signore è il suo nome. Alleluia!*} \volte{2}

\spazio

\strofa Voglio cantare in onore del Signore,\\
perché ha trionfato, alleluia!*\\
Ha gettato in mare cavallo e cavaliere,\\
mia forza e mio canto è il Signore.\\
Il mio Salvatore è il Dio di mio padre\\
ed io lo voglio esaltare.

\spazio

\rit{Cantiamo al Signore, stupenda è la sua vittoria...}

\spazio

\strofa Dio è prode in guerra, si chiama Signore,\\
travolse nel mare gli eserciti.\\
I carri d'Egitto sommerse nel Mar Rosso;\\
abissi profondi li coprono.\\
La tua destra Signore, si è innalzata,\\
la tua potenza è terribile.

\spazio

\rit{Cantiamo al Signore, stupenda è la sua vittoria...}

\spazio

\strofa Si accumularon le acque al tuo soffio,\\
s'alzaron le onde come un argine.\\
Si raggelaron gli abissi in fondo al mare.\\
Chi è come Te, o Signore?\\
Guidasti con forza il popolo redento\\
e lo conducesti verso Sion.

\spazio

\rit{Cantiamo al Signore, stupenda è la sua vittoria...}

\spazio

\emph{*Nel tempo di Quaresima la parola alleluia viene sostituita con: è gioia.}
