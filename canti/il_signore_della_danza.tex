\canto{Il Signore della danza}

\rit{Danza allor, dovunque tu sarai,\\
sono il Signore della danza sai;\\
e ti condurrò dovunque tu vorrai:\\
e per sempre nell'anima, tu danzerai.}

\spazio

\strofa Danzai al mattino quando tutto cominciò,\\
nel sole e nella luna il mio Spirito danzò;\\
Son sceso dal cielo per portar la verità:\\
e perciò chi mi segue sempre danzerà.

\spazio

\rit{Danza allor, dovunque tu sarai...}

\spazio

\strofa Danzai allora per gli scribi e i farisei,\\
ma erano incapaci e non sapevano imparar.\\
Quando ai pescatori Io chiesi di danzar,\\
subito impararono e si misero a danzar.

\spazio

\rit{Danza allor, dovunque tu sarai...}

\spazio

\strofa Di sabato volevano impedirmi di danzar,\\
ad uno zoppo a vivere, a sorridere a danzar.\\
Poi mi inchiodarono al legno di una croce,\\
ma non riuscirono a togliermi la voce.

\spazio

\rit{Danza allor, dovunque tu sarai...}

\spazio

\strofa Il cielo si oscurò quando danzai il venerdì,\\
ma è difficile danzar così.\\
Nella tomba, pensano, più non danzerà,\\
ma io sono la danza che mai finirà.

\spazio

\rit{Danza allor, dovunque tu sarai...}

\spazio

\strofa Sì sono vivo e continuo a danzar,\\
a soffrire, morire e ogni dì resuscitar.\\
Se vivrai in Me, Io vivrò in te,\\
ed allora vieni e danza insieme a Me.

\spazio

\rit{Danza allor, dovunque tu sarai...}

\spazio

\strofa Se mi presti il tuo corpo io danzerò in te,\\
perché la gioia è di gioire in te.\\
Quassù nel cielo non si suda più,\\
ma Io voglio stancarmi e scendo ancora giù.

\spazio

\rit{Danza allor, dovunque tu sarai...}