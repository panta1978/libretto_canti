\canto{Gesù Signore}

\strofa Santo mistero di luce e di grazia,\\
che ci dischiudi la strada del cielo,\\
vittima sacra che doni salvezza,\\
lieta la Chiesa ti canta e ti onora.\\
È questo rito la Pasqua perenne,\\
che c'incammina al traguardo del Regno.

\spazio

\rit{Gesù Signore che gli uomini nutri\\
della tua carne vera e del tuo sangue,\\
altro nome non c'è che sotto il cielo,\\
da colpa e morte ci possa salvare.}

\spazio

\strofa O pellegrino che bussi alla porta,\\
fa che t'apriamo solleciti il cuore,\\
Tu con Te rechi e cortese ci doni\\
il pane santo che dà vita eterna.\\
Ascolteremo la cara tua voce\\
e a tu per tu noi ceneremo insieme.

\spazio

\rit{Gesù Signore che gli uomini nutri...}

\spazio

\strofa O crocifisso Signore, il tuo sangue,\\
che sotto il segno del vino adoriamo,\\
il patto nuovo ed eterno sigilla:\\
tutti ci lava, riscatta e raduna.\\
Qui la speranza dell'uomo rinasce,\\
qui c'è la fonte di vita immortale.

\spazio

\rit{Gesù Signore che gli uomini nutri...}

\spazio

\strofa Figlio del re, che alle nozze tue inviti:\\
i derelitti, i mendichi, gli oppressi,\\
umili e grati alla festa veniamo:\\
al tuo banchetto fa posto anche a noi.\\
Della tua veste splendente vestiti,\\
nella tua casa sereni restiamo.

\spazio

\rit{Gesù Signore che gli uomini nutri...}

\spazio

\strofa Da quest'altare l'agnello immolato,\\
ringiovanisce la santa nazione:\\
tratta dal fianco del più vero Adamo,\\
la bella sposa ci pasce e rivive.\\
Di qui la Chiesa riceve vigore\\
di proclamare il Vangelo alle genti.

\spazio

\rit{Gesù Signore che gli uomini nutri...}