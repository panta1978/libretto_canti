\canto{Andrò a vederla un dì}

\strofa Andrò a vederla un dì in cielo patria mia;\\
andrò a veder Maria mia gioia e mio amor.

\spazio

\rit{Al ciel, al ciel, al ciel.\\
Andrò a vederla un dì.} \volte{2}

\spazio

\strofa Andrò a vederla un dì è il grido di speranza,\\
che infondemi costanza, nel viaggio fra i dolor.

\spazio

\rit{Al ciel, al ciel, al ciel...}

\spazio

\strofa Andrò a vederla un dì andrò a levar miei canti,\\
cogli angeli e coi Santi per corteggiarla ognor.

\spazio

\rit{Al ciel, al ciel, al ciel...}

\spazio

\strofa Andrò a vederla un dì le andrò vicino al trono,\\
ad ottener in dono, un serto di splendor.

\spazio

\rit{Al ciel, al ciel, al ciel...}

\spazio

\strofa Andrò a vederla un dì la vergine immortale;\\
m'aggirerò sull'ale, dicendole il mio amor.

\spazio

\rit{Al ciel, al ciel, al ciel...}

\spazio

\strofa Andrò a vederla un dì lasciando quest'esilio,\\
le poserò qual figlio il capo sopra il cor.

\spazio

\rit{Al ciel, al ciel, al ciel...}

\spazio

\strofa Andrò a vederla un dì sol lei mio core implora,\\
ma non la veggo ancora: è in cielo col Signor.

\spazio

\rit{Al ciel, al ciel, al ciel...}